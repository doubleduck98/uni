\documentclass[12pt]{article}
\usepackage[polish]{babel}
\usepackage{polski}
\usepackage[utf8]{inputenc}
\usepackage{graphicx} 
\usepackage{hyperref}\hypersetup{
    colorlinks=true,
    linkcolor=blue,
    filecolor=magenta,      
    urlcolor=blue,
}
\frenchspacing
 \begin{document}
 \begin{titlepage}
 \begin{center}
        \vspace*{1cm}
 
        \Huge
        \textbf{Złe interfejsy}
 
        \vspace{0.5cm}
        \LARGE
       
 
        \vspace{1.5cm}
 
        \textbf{Jakub Pacierpnik, Dawid Nadolski, Szymon Zienkiewicz}
 
        \vfill
    \end{center}
 \end{titlepage}
 \section{Źle funkcjonujące czujniki}
Przykładem źle zaprojektowanego interfejsu, z którym zetknąłem się w swoim otoczeniu, są nieprawidłowo umiejscowione czujniki ruchu odpowiedzialne za włączanie światła w garażu podziemnym. Po wejściu do garażu przez około trzydzieści sekund panuje całkowita ciemność, nawet po przebyciu znacznego dystansu, co będzie widać na zamieszczonym poniżej filmie. Jest to sytuacja nie do przyjęcia, która stwarza poważne zagrożenie dla zdrowia użytkowników garażu. Problemem jest tutaj odległość czujników od wejścia. W garażu sąsiednim są one rozmieszczone prawidłowo, przez co światło zostaje włączone praktycznie od razu po wejściu do środka, zapewniając dobrą widoczność i poczucie bezpieczeństwa.
 \newline
 \begin{center}
 \href{https://drive.google.com/open?id=1VdbMeJAAv0NQNLKfLPST7r1uHFxxe-jP}{Tutaj znajduje się link do filmu, na którym pokazany jest problem.}
 \end{center}
 \section{Płyta ursynowskiego rapera Pezeta}
 Płyty kompaktowe często ukazują się w dodatkowej kartonowej nakładce, tak było również w tym przypadku. Niestety wytwórnia, która zajmowała się produkcją tych opakowań nie zauważyła, że płytę bardzo trudno z tej nakładki wyciągnąć. Po bokach nakładki zostały wycięte rowki, które jak zgaduję powinny służyć do złapania za płytę i wyciągnięcie jej. Rowki te jednak są za małe i wyciągnięcie za ich pomocą płyty graniczy z cudem. Pozostają nam zatem dwie opcje wyciągnięcia płyty. Pierwszą z nich jest odchylenie nakładki na środku, złapanie za płytę oraz jej wyciągnięcie. Na swojej drodze napotkamy dwa problemy: po pierwsze płytę takim zabiegiem wciąż jest bardzo ciężko wyciągnąć małym nakładem sił, po drugie nakładka po paru takich próbach nosi już widoczne ślady użytkowania. Druga opcja jest najbardziej efektywna, mimo to wciąż niesie ze sobą sporo niebezpieczeństwa gdyż złapanie za nakładkę i wstrząsanie aż płyta się z niej wysunie grozi niekontrolowanym wypadnięciem i roztrzaskaniem się o podłogę.
 \newpage
 \begin{center}
 	\includegraphics[width=0.45\textwidth]{nadol1}\newline
 	1.Rowki do wyciągania płyty\newline\newline
 	\includegraphics[width=0.45\textwidth]{nadol2}\newline 
 	2.Zniszczenia po wyciąganiu płyty.\newline
 \end{center}
 \newpage
 \section{Menu ustawień kart graficznych AMD}
Kolejnym przykładem jest menu ustawień kart graficznych Radeon. Prezentuje się on całkiem dobrze, jest nowoczesny i miły dla oka, jednak walor estetyczny to nie jest główna cecha jaką powinien prezentować ten interfejs.
 \newline
 \begin{center}
 \includegraphics[width=1\textwidth]{1.png}
\end{center}
Gdy uruchamiamy menu, na samej górze znajdują się zakładki ustawień. W ich środkowej części znajdują się zakładki dla ustawień dla dodatkowych programów niedołączonych ze sterownikiem, co jest raczej niepotrzebne. Obok nich, znajdują się zakładki dla ustawień WIDEO i EKRANU, które często ze sobą myliłem. W zakładce WIDEO znajdują się jedynie predefiniowane zestawy kolorów, a w zakładce EKRAN cała reszta ustawień, także tych dotyczących barwy koloru, kontrastu itp. Dalej, jeżeli chcielibyśmy dowiedzieć się, co oznaczają poszczególne pola ustawień, możemy kliknąć przycisk więcej, ale niestety naszym oczom ukazuje się ściana tekstu, z której sami musimy sobie wybrać fragment, który opisuje interesujące nas pole.
\newline
\begin{center}
\includegraphics[width=1\textwidth]{5.png}
\end{center} Jest to bardzo nieczytelne i nieporęczne. Oprócz tego jest wiele innych mniejszych niedociągnięć takich jak brak instrukcji w menu zmieniania rozdzielczości, albo brak ustawień kolorów w menu nakładek na programy. Sposobem na poprawienie interfejsu jest albo gruntowne poprawienie czytelności, albo oferowanie prostego menu graficznego, bez ładnego tła, ale za to szybkiego i intuicyjnego (zupełnie tak jak konkurencyjna nVidia).
 \end{document}

 